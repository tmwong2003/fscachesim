% $Header: $

\documentclass[twoside]{article}

\usepackage[small,bf]{caption}
\usepackage{fullpage}
\usepackage{hyperref}
\usepackage{times}
\usepackage{url}

%
%
%

\newcommand{\fscachesim}{\texttt{fscachesim}}

\newcommand{\option}[2]{\texttt{#1}\textit{#2}}

%
%
%

\title{\fscachesim: A quick-and-dirty client-array cache simulator}

\author{Ted Wong}

\date{}

%
%
%

\begin{document}

\maketitle

%
%
%

This manual describes how to use \fscachesim, a quick-and-dirty simulator I
wrote to model the behavior of (large) client and array caches that
implement various experimental management protocols. In particular, the
simulator models protocols that use the \textsc{demote} operation proposed
by myself and John Wilkes \cite{Wong2000} for transferring read-only disk
blocks from client to array caches.

%
%
%

\section{Command summary}

\fscachesim{}\quad \textit{option-flags...\quad client-cache-size\quad
array-cache-size\quad trace-files...}

\subsection{Option flags}

\begin{description}

\item[\option{-b}{\quad block-size}] \quad\\ Specify the block size in
bytes. The default is 4096.

\item[\option{-D}] \quad\\ Enable demotion of blocks from the client to the
array. The array will cache read and demoted blocks in LRU order unless you
enable ghost caches with \option{-g}.

\item[\option{-d}] \quad\\ Enable demotion of blocks from the client to the
array. The array will cache read blocks in MRU order, and demoted blocks in
LRU order.

\item[\option{-g}] \quad\\ Enable ghost caches and segmented
cache insertion at the array. You must also enable demotions (with
\option{-D}) explicitly.

\item[\option{-m}] \quad\\ Read MAMBO-format trace files \cite{Uysal1997}.

\item[\option{-s}{\quad probationary-cache-size}] \quad\\ Enable a
segmented LRU cache at the array \cite{Karedla1994} with a probationary
cache size of \option{}{probabtionary-cache-size} MB.

\item[\option{-w}{\quad warmup-time}] \quad\\ Specify the warm-up
time in seconds. After the warm-up time elapses, \fscachesim{} will reset
all cache statistics, but retain the cache contents. The default is 0.

\end{description}

\subsection{Parameters}

\begin{description}

\item[\textit{client-size-size}] \quad\\ Specify the per-client cache size in
MB. All clients have the same cache size.

\item[\textit{array-size-size}] \quad\\ Specify the array cache size in MB.

\item[\textit{trace-files...}] \quad\\ Specify the trace files. \fscachesim{}
creates a simulated client for each input trace file.

\end{description}

\section{Input trace file format}

\fscachesim{} uses plain-text input trace files. Each line in a file
corresponds to a single I/O request, and has the following format (where
\textit{time-request-issued} is in seconds, and \textit{offset-into-object} and
\textit{length-of-request} are in bytes):

\begin{displaymath}
\mbox{$\textit{time-request-issued\quad object-ID\quad offset-into-object\quad
length-of-request}$}
\end{displaymath}

\fscachesim{} can also use MAMBO input trace files \cite{Uysal1997}.

\section{Example usage}

\begin{description}

\item[\texttt{\fscachesim{} -w 32768 -D 64 64 random}] \quad\\ Simulate a
64 MB client and a 64 MB array system. The client will demote blocks, and
the array will cache read and demoted blocks in LRU order. The client will
replay I/O requests from the file named `\texttt{random}'. \fscachesim{}
will reset the cache statistics after 32768 seconds of simulated time (as
measured by the value of \textit{time-request-issued} on the requests).

\item[\texttt{\fscachesim{} -w 32768 -d 64 64 random}] \quad\\ Simulate a
64 MB client and a 64 MB array. The client will demote blocks, and the
array will cache read blocks in MRU order, and demoted blocks in LRU
order. The client will replay I/O requests from
`\texttt{random}'. \fscachesim{} will reset the cache statistics after
32768 seconds of simulated time (as measured by the value of
\textit{time-request-issued} on the requests).

\item[\texttt{\fscachesim{} -g 256 2048 db2-1 db2-2}] \quad\\ Simulate
two 256 MB clients and a 2 GB (2048 MB) array. The client will discard
blocks that it ejects from its cache. The array will insert read blocks
into the cache based on the read ghost cache score. One client will replay
I/O requests from `\texttt{db2-1}', and the other client will replay
requests from `\texttt{db2-2}'.

\item[\texttt{\fscachesim{} -g -D 256 2048 httpd-1 httpd-2}] \quad\\ Simulate
two 256 MB clients and a 2 GB (2048 MB) array. The clients will demote
blocks, and the array will insert read and demoted blocks into the cache
based on the read and demote ghost cache scores. One client will replay I/O
requests from `\texttt{httpd-1}', and the other client will replay requests
from `\texttt{httpd-2}'.

\end{description}

\bibliography{biblio}
\bibliographystyle{abbrv}

\end{document}