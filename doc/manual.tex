% $Header: /afs/cs.cmu.edu/user/tmwong/Cvs/fscachesim/doc/manual.tex,v 1.1 2001/11/14 22:07:59 tmwong Exp $

\documentclass[twoside]{article}

\usepackage[small,bf]{caption}
\usepackage{fullpage}
\usepackage{hyperref}
\usepackage{times}
\usepackage{url}

%
%
%

\newcommand{\fscachesim}{\texttt{fscachesim}}

\newcommand{\option}[2]{\texttt{#1}\textit{#2}}

%
%
%

\title{\fscachesim: A quick-and-dirty client-array cache simulator}

\author{Ted Wong}

\date{}

%
%
%

\begin{document}

\maketitle

%
%
%

This manual describes how to use \fscachesim, a quick-and-dirty simulator I
wrote to model the behavior of (large) client and array caches that
implement various experimental management protocols. In particular, the
simulator models protocols that use the \textsc{demote} operation proposed
by myself and John Wilkes \cite{Wong2002} for transferring read-only disk
blocks from client to array caches.

%
%
%

\section{Command summary}

\fscachesim{}\quad \textit{option-flags...\quad array-type\quad
client-size\quad array-size\quad trace-files...}

\subsection{Option flags}

\begin{description}

\item[\option{-b}{\quad block-size}] \quad\\ Specify the block size in
bytes. The default is 4096.

\item[\option{-c}{\quad warmup-count}] \quad\\ Specify the warm-up
I/O count. After processing the warm-up I/O count, \fscachesim{} will reset
all cache statistics, but retain the cache contents. The default is 0.

\item[\option{-d}] \quad\\ Enable demotion of blocks from the client to the
array.

\item[\option{-m}] \quad\\ Read MAMBO-format trace files \cite{Uysal1997}.

\item[\option{-w}{\quad warmup-time}] \quad\\ Specify the warm-up
time in seconds. After the warm-up time elapses, \fscachesim{} will reset
all cache statistics, but retain the cache contents. The default is 0.

\end{description}

\subsection{Parameters}

\begin{description}

\item[\textit{array-type}] \quad\\ Specify the array type. \fscachesim{}
supports a number of cache models:

LRU: Cache blocks in traditional LRU order.

MRULRU: Cache blocks read from disk in MRU order, and blocks demoted from
clients in LRU order.

RSEGEXP: Cache blocks using the adaptive caching models described in
the USENIX 2002 paper \cite{Wong2002}, in which we divide the ejection
queue into ten exponential size segments. Each cache segment is twice the
size of the previous segment, and the largest segment is at the end of the
queue. We maintain \emph{ghost caches}, which measure how often read
requests would hit in the actual cache if we only cached either read or
demoted blocks. We then insert blocks read from disk (and blocks demoted
from clients) into a segment in the actual cache based on the hit rate in
the read (or demote) ghost.

\begin{displaymath}
\mathit{in\mbox{-}seg}_r =
	\frac{\mathit{hits}_r}{\mathit{hits}_r + \mathit{hits}_d} \mathit{segs}
\quad\mbox{and}\quad
\mathit{in\mbox{-}seg}_d =
	\frac{\mathit{hits}_d}{\mathit{hits}_r + \mathit{hits}_d} \mathit{segs}
\end{displaymath}

NSEGEXP: Similar to RSEGEXP, except that we normalize the insertion segment
number by the maximum of the two potential numbers.

\{N,R\}SEGUNI: Similar to \{N,R\}SEGEXP, except with uniform size segments.

\item[\textit{client-size}] \quad\\ Specify the per-client cache size in
MB. All clients have the same cache size.

\item[\textit{array-size}] \quad\\ Specify the array cache size in MB.

\item[\textit{trace-files...}] \quad\\ Specify the trace files. \fscachesim{}
creates a simulated client for each input trace file.

\end{description}

\section{Input trace file format}

\fscachesim{} uses plain-text input trace files. Each line in a file
corresponds to a single I/O request, and has the following format (where
\textit{time-request-issued} is in seconds, and \textit{offset-into-object} and
\textit{length-of-request} are in bytes):

\begin{displaymath}
\mbox{$\textit{time-request-issued\quad object-ID\quad offset-into-object\quad
length-of-request}$}
\end{displaymath}

\fscachesim{} can also use MAMBO input trace files \cite{Uysal1997}.

\section{Example usage}

\begin{description}

\item[\texttt{\fscachesim{} -w 32768 -d LRU 64 64 random}] \quad\\ Simulate a
64 MB client and a 64 MB array system. The client will demote blocks, and
the array will cache read and demoted blocks in LRU order. The client will
replay I/O requests from the file named `\texttt{random}'. \fscachesim{}
will reset the cache statistics after 32768 seconds of simulated time (as
measured by the value of \textit{time-request-issued} on the requests).

\item[\texttt{\fscachesim{} -w 32768 -d MRULRU 64 64 random}] \quad\\
Simulate a 64 MB client and a 64 MB array. The client will demote blocks,
and the array will cache read blocks in MRU order, and demoted blocks in
LRU order. The client will replay I/O requests from
`\texttt{random}'. \fscachesim{} will reset the cache statistics after
32768 seconds of simulated time (as measured by the value of
\textit{time-request-issued} on the requests).

\item[\texttt{\fscachesim{} RSEGEXP 256 2048 db2-1 db2-2}] \quad\\ Simulate
two 256 MB clients and a 2 GB (2048 MB) array. The client will discard
blocks that it ejects from its cache. The array will insert read blocks
into the cache based on the read ghost cache score. One client will replay
I/O requests from `\texttt{db2-1}', and the other client will replay
requests from `\texttt{db2-2}'.

\item[\texttt{\fscachesim{} -d RSEGEXP 256 2048 httpd-1 httpd-2}] \quad\\
Simulate two 256 MB clients and a 2 GB (2048 MB) array. The clients will
demote blocks, and the array will insert read and demoted blocks into the
cache based on the read and demote ghost cache scores. One client will
replay I/O requests from `\texttt{httpd-1}', and the other client will
replay requests from `\texttt{httpd-2}'.

\end{description}

\begin{thebibliography}{UAS97}

\bibitem[UAS97]{Uysal1997}
Mustafa Uysal, Anurag Acharya, and Joel Saltz.
\newblock Requirements of {I/O} systems for parallel machines: An
  application-driven study.
\newblock Technical Report CS-TR-3802, Dept. of Computer Science, University of
  Maryland, College Park, MD, May 1997.

\bibitem[WW02]{Wong2002}
Theodore M. Wong and John Wilkes.
\newblock My cache or yours? Making storage more exclusive,
\newblock in Proc. of the 2002 USENIX Annual Technical Conference (June 2002)

\end{thebibliography}

\end{document}